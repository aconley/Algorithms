\topglue 0.5in
\centerline{Notes on Knuth Chapter 1}
\vskip 0.5in

\noindent\centerline{{\bf Section 1.3.3: Applications to Permutations}}
\vskip 0.2in
\noindent {\bf Algorithm A} A key point about Algorithm A is that the step {\bf A4},
{\bf A5} combo wraps entirely.  So in many cases {\tt CURRENT} will
re-encounter itself.  This is actually critical for the final occurance of
any symbol -- it has to be able to find the next element.  

For example,
consider $\left(a c f\right)\left(b d\right)$.  Pretty quickly, one gets
{\tt START} $\gets a$, {\tt CURRENT} $\gets c$.  But there is only one $c$
in the formula, so what happens is that the code progresses to the right,
hits the end, outputs $c$, and then wraps around until it runs into the 
{\it same} $c$.  It then tags that $c$ and sets {\tt CURRENT} $\gets f$.

\vskip 0.15in
\noindent {\bf Algorithm B} Try a different (slightly simpler) example:
the product of $\left(a b d\right) \left(e f\right)$ and 
$\left(a c f\right)\left(b d\right)$ (this comes from problems 3 and 4
of the exercises).  We have $n = 6$, so we need $T\left[1\right] \ldots T\left[6\right]$.
\vskip 0.05in
\noindent {\bf B1:} $T \gets {1, 2, 3, 4, 5, 6}$. \hfil\break
\noindent {\bf B2:} The element is ``)", so $Z \gets 0$ and repeat.\hfil\break
\noindent {\bf B2:} The element is ``d", corresponding to $x_4$, so $i = 4$.\hfil\break
\noindent {\bf B3:} Exchange $Z$ and $T\left[4\right]$, so we now have
 $T = {1, 2, 3, 0, 5, 6}$ and $Z = 4$.  We have made $T\left[i\right] = 0$,
  so $j \gets 4$.\hfil\break
\noindent {\bf B2:} The next element is ``b", and so $i \gets 2$.\hfil\break
\noindent {\bf B3:} Exchange $Z$ and $T\left[2\right]$, so that
 $Z = 2$ and $T = {1, 4, 3, 0, 5, 6}$.\hfil\break
\noindent {\bf B2:} The next element is ``(", so jump to {\bf B4}.\hfil\break
\noindent {\bf B4:} Set $T\left[j\right] \gets Z$, so $T = {1, 4, 3, 2, 5, 6}$.\hfil\break
\noindent {\bf B2:} It's a ``)" so $Z \gets 0$.\hfil\break
\noindent {\bf B2:} It's a ``f'', so $i \gets 6$.\hfil\break
\noindent {\bf B3:} Exchange $Z$ and $T\left[6\right]$, giving
 $T = {1, 4, 3, 2, 5, 0}$, $Z = 6$, and set $j \gets 6$.\hfil\break
\noindent {\bf B2:} It's a ``c", so $i \gets 3$.\hfil\break
\noindent {\bf B3:} $Z \gets 3$, $T = {1, 4, 6, 2, 5, 0}$.\hfil\break
\noindent {\bf B2:} It's a ``a", so $i \gets 1$\hfil\break
\noindent {\bf B3:} $Z \gets 1$, $T = {3, 4, 6, 2, 5, 0}$.\hfil\break
\noindent {\bf B2:} It's a ``(", so on to {\bf B4}.\hfil\break
\noindent {\bf B4:} Now set $T\left[6\right] \gets Z$, so
 $T = {3, 4, 6, 2, 5, 1}.$
At this point we have written $\left(a c f\right)\left( b d \right)$ into $T$
as $a \to c$, $b \to d$, $c \to f$, $d \to b$, $e \to e$, $f \to a$ -- which
is correct. Now we will start the actual product. \hfil\break
\noindent {\bf B2:} $Z \gets 0$, then repeat to get $i \gets 6$.\hfil\break
\noindent {\bf B3:} $Z = 1$, $T = 3, 4, 6, 2, 5, 0$ and $j \gets 6$.\hfil\break
\noindent {\bf B2, B3:} $i \gets 5$, $Z \gets 5$, $T = 3, 4, 6, 2, 1, 0$.\hfil\break
\noindent {\bf B2:} It's a ``('', so to {\bf B4}.\hfil\break
\noindent {\bf B4:} $T = 3, 4, 6, 2, 1, 5$.\hfil\break
\noindent {\bf B2, B2, B3:} $Z \gets 0$, then $i \gets 4$, then $T = 3, 4, 6, 0, 1, 5$
and $j \gets 4$ and $Z \gets 2$.\hfil\break
\noindent {\bf B2, B3:} $i \gets 2$, $Z \gets 4$, $T = 3, 2, 6, 0, 1, 5$.\hfil\break
\noindent {\bf B2, B3:} $i \gets 1$, $Z \gets 3$, $T = 4, 2, 6, 0, 1, 5$.\hfil\break
\noindent {\bf B4:} Finally, $T\left[j\right] \gets Z$ for the last modification, giving
$T = 4, 2, 6, 3, 1, 5$, or $a \to d$, $b \to b$, $c \to f$, $d \to c$, $e \to a$, and
$f \to e$, which is correct.

Indeed, at every point as we sweep from right to left, $T$ represents all the
product to the right of the current point.

\end