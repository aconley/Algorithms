\topglue 0.5in
\centerline{Notes on Knuth Chapter 1}
\vskip 0.5in

\noindent\centerline{{\bf Section 1.3.3: Applications to Permutations}}
\vskip 0.2in
\noindent {\bf Algorithm A} ({\it Multiplication of Permutations}).
A key point about Algorithm A is that the step {\bf A4},
{\bf A5} combo wraps entirely.  So in many cases {\tt CURRENT} will
re-encounter itself.  This is actually critical for the final occurance of
any symbol -- it has to be able to find the next element.  

For example,
consider $\left(a c f\right)\left(b d\right)$.  Pretty quickly, one gets
{\tt START} $\gets a$, {\tt CURRENT} $\gets c$.  But there is only one $c$
in the formula, so what happens is that the code progresses to the right,
hits the end, outputs $c$, and then wraps around until it runs into the 
{\it same} $c$.  It then tags that $c$ and sets {\tt CURRENT} $\gets f$.

\vskip 0.15in
\noindent {\bf Algorithm B} ({\it Multiplication of Permutations})
Try a different (slightly simpler) example:
the product of $\left(a b d\right) \left(e f\right)$ and 
$\left(a c f\right)\left(b d\right)$ (this comes from problems 3 and 4
of the exercises).  We have $n = 6$, so we need $T\left[1\right] \ldots T\left[6\right]$.
\vskip 0.05in
\noindent {\bf B1:} $T \gets {1, 2, 3, 4, 5, 6}$. \hfil\break
\noindent {\bf B2:} The element is ``)", so $Z \gets 0$ and repeat.\hfil\break
\noindent {\bf B2:} The element is ``d", corresponding to $x_4$, so $i = 4$.\hfil\break
\noindent {\bf B3:} Exchange $Z$ and $T\left[4\right]$, so we now have
 $T = {1, 2, 3, 0, 5, 6}$ and $Z = 4$.  We have made $T\left[i\right] = 0$,
  so $j \gets 4$.\hfil\break
\noindent {\bf B2:} The next element is ``b", and so $i \gets 2$.\hfil\break
\noindent {\bf B3:} Exchange $Z$ and $T\left[2\right]$, so that
 $Z = 2$ and $T = {1, 4, 3, 0, 5, 6}$.\hfil\break
\noindent {\bf B2:} The next element is ``(", so jump to {\bf B4}.\hfil\break
\noindent {\bf B4:} Set $T\left[j\right] \gets Z$, so $T = {1, 4, 3, 2, 5, 6}$.\hfil\break
\noindent {\bf B2:} It's a ``)" so $Z \gets 0$.\hfil\break
\noindent {\bf B2:} It's a ``f'', so $i \gets 6$.\hfil\break
\noindent {\bf B3:} Exchange $Z$ and $T\left[6\right]$, giving
 $T = {1, 4, 3, 2, 5, 0}$, $Z = 6$, and set $j \gets 6$.\hfil\break
\noindent {\bf B2:} It's a ``c", so $i \gets 3$.\hfil\break
\noindent {\bf B3:} $Z \gets 3$, $T = {1, 4, 6, 2, 5, 0}$.\hfil\break
\noindent {\bf B2:} It's a ``a", so $i \gets 1$\hfil\break
\noindent {\bf B3:} $Z \gets 1$, $T = {3, 4, 6, 2, 5, 0}$.\hfil\break
\noindent {\bf B2:} It's a ``(", so on to {\bf B4}.\hfil\break
\noindent {\bf B4:} Now set $T\left[6\right] \gets Z$, so
 $T = {3, 4, 6, 2, 5, 1}.$
At this point we have written $\left(a c f\right)\left( b d \right)$ into $T$
as $a \to c$, $b \to d$, $c \to f$, $d \to b$, $e \to e$, $f \to a$ -- which
is correct. Now we will start the actual product. \hfil\break
\noindent {\bf B2:} $Z \gets 0$, then repeat to get $i \gets 6$.\hfil\break
\noindent {\bf B3:} $Z = 1$, $T = 3, 4, 6, 2, 5, 0$ and $j \gets 6$.\hfil\break
\noindent {\bf B2, B3:} $i \gets 5$, $Z \gets 5$, $T = 3, 4, 6, 2, 1, 0$.\hfil\break
\noindent {\bf B2:} It's a ``('', so to {\bf B4}.\hfil\break
\noindent {\bf B4:} $T = 3, 4, 6, 2, 1, 5$.\hfil\break
\noindent {\bf B2, B2, B3:} $Z \gets 0$, then $i \gets 4$, then $T = 3, 4, 6, 0, 1, 5$
and $j \gets 4$ and $Z \gets 2$.\hfil\break
\noindent {\bf B2, B3:} $i \gets 2$, $Z \gets 4$, $T = 3, 2, 6, 0, 1, 5$.\hfil\break
\noindent {\bf B2, B3:} $i \gets 1$, $Z \gets 3$, $T = 4, 2, 6, 0, 1, 5$.\hfil\break
\noindent {\bf B4:} Finally, $T\left[j\right] \gets Z$ for the last modification, giving
$T = 4, 2, 6, 3, 1, 5$, or $a \to d$, $b \to b$, $c \to f$, $d \to c$, $e \to a$, and
$f \to e$, which is correct.

Indeed, at every point as we sweep from right to left, $T$ represents all the
product to the right of the current point.

\vfil\break
\topglue 0.5in
\centerline{Notes on Knuth Chapter 2}
\vskip 0.3in

\noindent\centerline{{\bf Section 2.2.4: Circular Lists}}
\vskip 0.2in

\noindent {\bf Algorithm A} ({\it Addition of Polynomials}).  Doing the suggested example,
which is $x + y - z$ added to $x^2 - 2 y - z$.  The first ({\tt P}) is represented
by a list with {\tt COEF}s 1, 1, 1, and {\tt ABC} values 100, 010, 001.  The second
has {\tt COEF}s 1, -2, -1, and {\tt ABC} values 200, 010, 001.

So, after step {\bf A1}, we will have {\tt P} pointing at the $x$ term, {\tt Q} pointing
at the $x^2$ term, and {\tt Q1} on the terminator node.  At {\bf A2} we have
${\tt ABC(P)} < {\tt ABC(Q)}$, with values 100 and 200, respectively, so move the
{\tt Q} pointers along so that {\tt Q} points at {\tt COEF} 010 and {\tt Q1} at the
$x^2$ term.  What is happening is that {\tt P} doesn't have a $x^2$ term, so
we are just leaving the one in {\tt Q} alone and moving along.

  Now in {\bf A2} we have ${\tt ABC(P)} > {\tt ABC(Q)}$ (since
the latter is now 010, so we go to {\bf A5}.  This is adding a $x$ term to {\tt Q},
which doesn't have one.  So me make a new node {\tt Q2}, set it's {\tt COEF}
to 1, and it's {\tt ABC} to 010.  We then juggle pointers so that {\tt P} gets
moved along one to $+y$, and {\tt Q} now includes $+x$, and the {\tt Q}
pointers are moved along one (so that {\tt Q} still points at $-2y$ but {\tt Q1}
points at the new node).

Now we have both {\tt P} and {\tt Q} pointing at a {\tt COEF} 010 term ($y$),
so we go to {\bf A3} and add, updating the coefficient so that {\tt Q} now
has {\tt COEF} -1 for $y$.  Then we move {\tt P} along to $+z$, {\tt Q}
along to $-z$, and {\tt Q1} to the $y$ term.

Again they match, but this time when we add we get {\tt COEF} 0 and so go to
{\bf A4}.  We remove the node by making the {\tt LINK} of {\tt Q1} (which is
parked on $y$) point to {\tt LINK(Q)}, which points at the terminal node,
and have {\tt Q} point to the terminal node.  {\tt P} is also moved along, also
to its terminal node.  Since {\tt P} and {\tt Q} are both at their terminal node,
we end up at {\bf A3} with ${\tt ABC(P)} < 0$, and so we terminate.  
The sum is $x^2 + x - y$.

\vskip 0.3in
\noindent\centerline{{\bf Section 2.3.1: Traversing Binary Trees}}
\vskip 0.2in

\noindent {\bf Algorithm S} ({\it Inorder successor in a threaded binary tree}).
By definition, if {\tt RTAG(P) = 1} then {\tt RLINK(P)} points at {\tt \$P}.
That explains {\bf S1}, but what if it wasn't?  Well, the inorder successor
to {\tt P} is the leftmost child of {\tt RLINK(P)}.  So as long as we keep finding
actual left children (so, {\tt LTAG = 0} on the current node), just keep
following them.  As soon as we hit something without a real left link
(so, {\tt LTAG = 1}), that's the leftmost child, so we can stop.

\vskip 0.1in
\noindent {\bf Algorithm I} ({\it Insertion into a threaded binary tree}).
Lets see how this works on the threaded tree shown in (7), which
is also shown in (10).  

First, the simple case: imagine inserting a node K onto the already
existing node H.  H doesn't have any real links, so this should be simple.
So here {\tt P} is node H, {\tt Q} is the new node K.\hfil\break
\noindent {\bf I1:} ${\tt RLINK(K)} \gets {\tt F}$, ${\tt RTAG(K)} \gets 1$,
${\tt RLINK(H)} \gets {\tt K}$, ${\tt RTAG(H)} \gets {\tt 0}$, ${\tt LLINK(K)}
\gets {\tt H}$, ${\tt LTAG(K)} \gets {\tt 1}$. \hfil\break
\noindent {\bf I2:} {\tt RTAG(K)} is 1, so terminate.
\noindent and this is exactly what we want.  H now has a real right node,
and K has thread links with the left one pointing to it's predecessor (H),
and it's right link pointing to it's successor (F), both inorder.

Inserting in the middle is, of course, much more complicated.
Let's try inserting a node K between C and F.  Now:
\noindent {\bf I1:} ${\tt RLINK(K)} \gets {\tt F}$, ${\tt RTAG(K)} \gets 0$,
${\tt RLINK(C)} \gets {\tt K}$, ${\tt RTAG(H)} \gets {\tt 0}$, ${\tt LLINK(K)}
\gets {\tt C}$, ${\tt LTAG(K)} \gets {\tt 1}$. \hfil\break
\noindent {\bf I2:} Now we have {\tt RTAG(K)} of 0, so we need to find {\tt K\$}.
We do that by calling {\bf Algorithm S} with ${\tt P} \gets {\tt K}$.\hfil\nobreak
\noindent {\bf S1:} ${\tt Q} \gets {\tt F}$.\hfil\nobreak
\noindent {\bf S2:} ${\tt LTAG(F)} = 0$, so ${\tt Q} \gets {\tt H}$.\hfil\nobreak
\noindent {\bf S2:} ${\tt LTAG(H)} = 1$, so terminate, returning {\tt H}.\hfil\nobreak
So {\tt K\$} is {\tt H}, which is right -- the inorder search will go to the left subtree
of {\tt K}s only real child next, which is {\tt H}.  So, finally,
${\tt LLINK(H)} \gets {\tt L}$ -- that is, {\tt K} is {\tt H}s new successor.

So, it works, although it is obviously rather complicated, Knuth calling it simple
aside!

\end