\def\newstep#1{\smallskip \noindent {\bf #1}}
\def\newhead#1{\vskip 0.1in \noindent {\bf #1}}
\def\subsec#1{\vskip 0.08in \noindent {\bf #1}}
\def\sec#1 {\vfil\break \centerline{\tt #1} \vskip 0.2in}

\topglue 0.5in
\centerline{Notes on Knuth Fascicle 5: MPR, Backtracking, Dancing Links}
\vskip 0.3in
\centerline{\tt Mathematical Preliminaries Redux}
\vskip 0.2in

\noindent [p 3] Forumula (12): Expectation of expectations: show that
$E\left( E\left(X | Y\right)\right) = E\left(X\right)$.  As explained in the text, $E\left(X|Y\right)$
is a random function -- that is, a map from events $\omega \in \Omega$ to real numbers.   For an 
event $\omega$ it takes the value:
$$
  E\left(X|Y\right)\left(\omega\right) =
    \sum_{\omega^{\prime} \in \Omega} X\left(\omega^{\prime}\right)
      \left[Y\left(\omega\right) = Y\left(\omega^{\prime}\right)\right]
      {Pr\left(\omega^{\prime}\right) \over Pr \left(Y = Y\left(\omega\right)\right)}.
$$
where
$$
  Pr\left(Y = Y\left(\omega\right)\right) = \sum_{\omega^{prime} \in  \Omega}
    Pr \left(\omega^{\prime}\right) \left[ Y\left(\omega\right) = Y\left(\omega^{\prime}\right)\right]
$$
and $[]$ is the indicator function.  For example, if $\Omega = {-2, -1, 0, 1, 2}$, $Pr\left(\omega\right)
= 1/5$, and $Y\left(\omega\right) = \omega^2$, then 
$Pr \left(Y = 1\right) = Pr\left(1\right) + Pr\left(-1\right) = 2/5$.

So, continuing:
$$
\eqalign{
  E\left(E\left(X|Y\right)\right) =&\sum_{\omega \in \Omega} \sum_{\omega^{\prime} \in \Omega}
  X\left(\omega^{\prime}\right)
  \left[Y\left(\omega\right) = Y\left(\omega^{\prime}\right)\right]
  {Pr\left(\omega^{\prime}\right) Pr\left(\omega\right) \over Pr \left(Y = Y\left(\omega\right)\right)} \cr
   =& \sum_{\omega^{\prime} \in \Omega} \left(
     X\left(\omega^{\prime}\right) Pr\left(\omega^{\prime}\right) 
      \sum_{\omega \in \Omega} {\left[Y\left(\omega\right) = Y\left(\omega^{\prime}\right)\right]
       Pr\left(\omega\right) \over Pr \left(Y = Y\left(\omega\right)\right)} \right).
}
$$
Now, the inner sum is just one, since it is just the definition of $Pr\left(Y = Y\left(\omega\right)\right)$ summed
over all events $\omega$.  So this is just $E\left(E\left(X|Y\right)\right) = \sum_\omega X\left(\omega\right)
Pr \left(\omega\right) = E\left(X\right)$, as claimed.

\vskip 0.1in
\noindent {\bf Inequalities}
\vskip 0.05in

\noindent [p 5] Formula (23): derivation of the second moment principle.  The
final step again takes advantage of ${\rm E}\left(X | X > 0\right) = {\rm E} \left(X | X > 0\right)
{\rm Pr}\left(X > 0\right) + {\rm E} \left(X | X = 0\right) {\rm Pr}\left(X = 0\right) = {\rm E} \left(X | X > 0\right)
{\rm Pr}\left(X > 0\right)$, and therefore $\left( {\rm E} \left(X | X > 0\right) \right)^2 = 
\left({\rm E} \left(X \right) / {\rm Pr}\left(X > 0\right) \right)^2$.

\vskip 0.1in
\noindent {\bf Martingales}
\vskip 0.05in

\noindent Something that goes oddly unmentioned, but is used, is that
${\rm E} Z_n = {\rm E} Z_0$ for any martingale.  For example, it gets used in the stopping rule
discussion.  {\it Proof}: We have 
$$
\eqalign{ 
{\rm E}\, Z_n 
 &= \sum_{Z_{n-1}, \ldots, Z_{0}} {\rm E} \left(Z_n | Z_{n-1} \ldots, Z_0 \right)
                {\rm Pr}\left(Z_{n-1}, \ldots, Z_{0} \right)\cr 
 &= \sum_{Z_{n-1}, \ldots, Z_{0}} z_{n-1} {\rm Pr}\left(Z_{n-1}  \ldots, Z_{0} \right) \cr
 &= \sum_{Z_{n-1}, \ldots, Z_{0}} z_{n-1} {\rm Pr}\left(Z_{n-1} | Z_{n-2}, \ldots, Z_0\right)
                 {\rm Pr} \left(Z_{n-2}, \ldots, Z_0 \right) \cr
 &= {\rm E} Z_{n-1}.\cr
}
$$
And we can keep doing this all the way down to ${\rm E}\, Z_0$.

\sec {Basic Backtracking}

\noindent The way that backtracking is implemented does not really square with
modern programming pracitce.  Generally, we would probably want to implement
backtracking as some form of iterator.  Furthermore, the use of gotos is generally
frowned upon.  In most cases in Knuth, it's relatively easy to replace gotos
with if statements or for loops, but for Backtracking it is a bit more complicated
because the different paths are not nested, but instead interlock in an interesting
way.

This is easiest in Walker's approach, where we maintain the available values at
each level.  The iterator has three states: NEW, READY, and DONE.  A new iterator
starts in the NEW state.  The interesting stuff happens when we call Next.
Here I assume something like the Rust model, where we return None if there are no
more solutions, and Some(vector) for a solution.

\vskip 0.1in
\noindent {\bf WI1}\hfil\break 
\phantom{1}\hskip1em If the iterator state is DONE, return None.\hfil\break
\phantom{2}\hskip1em If the iterator state is READY, set $l \leftarrow l - 1.$\hfil\break
\phantom{3}\hskip1em If the iterator state is NEW, set $l \leftarrow 0$, compute $S_0$, and set
the state to READY.\hfil\break
\noindent {\bf WI2} Repeat this step while $S_l = \emptyset$: if $l = 0$ then set 
the state to DONE and return None.  Otherwise, $l \leftarrow l - 1$.
\vskip 0.05in
\noindent {\bf WI3} Set $x_l \leftarrow {\rm min}\, S_l$, $S_l \leftarrow S_l / x_l$, and 
$l \leftarrow l + 1$.
\vskip 0.05in
\noindent {\bf WI4} If $l = n$, then return ${\rm Some}\left(x_0, \ldots, x_{n-1}\right)$.
\vskip 0.05in
\noindent {\bf WI5} Compute $S_l$ and return to {\bf WI2}.
\vskip 0.1in

Unfortunately, this isn't as easy to do in the non-Walker approach.  In some
cases this can be avoided, but the general approach is to define some special
sentinel value $s_l$ at each level which is not part of ${\cal D}_l$, and which is
more than any valid value and which is the sucessor of the maximum valid ${\cal D}_l$.
For example, for Sudoku we could use 10 as a maximum sentinel, for Langford pairs
or for $n$-queens we could use $n+1$, etc.  Rather than using a maximum sentinel value, 
one can use a minimum instead.

A new iterator is created in the NEW state.

\vskip 0.1in
\noindent {\bf BI1}\hfil\break 
\phantom{1}\hskip1em If the iterator state is DONE, return None.\hfil\break
\phantom{2}\hskip1em If the iterator state is READY, set $l \leftarrow l - 1.$\hfil\break
\phantom{3}\hskip1em If the iterator state is NEW, set $l \leftarrow 0$, compute $S_0$, and set
the state to READY.\hfil\break
\noindent {\bf BI2} If $x_l = s_l$ then: if $l == 0$, set the state to DONE and return None.
  Otherwise, backtrack by setting $l \leftarrow l - 1$, downdate the datastructures appropriately,
  set $x_l \leftarrow {\rm next} {\cal D}_l$, and repeat this step.
\vskip 0.05in
\noindent {\bf BI3} If $P\left(x_0, \ldots, x_l\right)$ then set $l \leftarrow l + 1$.  If
  $l = n$ than return ${\rm Some}\left(x_0, \ldots, x_{n-1}\right)$.  Otherwise
  update the datastructures for the new level, set $x_l \leftarrow {\rm min} {\cal D}_l$,
  and return to {\bf BI2}.
\noindent {\bf BI4} Set $x_l \leftarrow {\rm next} {\cal D}_l$ and return to {\bf BI2}.
\vskip 0.1in

\noindent I implemented the N-Queens algorithms, including the optimizations
discussed in the problems, in c++ and timed their performance using
Google Benchmark.  The four methods were a  unoptimized
version, the array version of the basic algorithm~B, the
same version modified to use bit twiddling instead of arrays,
and Walkers version.  The execution time runs in the ratio
22:6.3:6.6:1.  So bit twiddling is not helpful, but, unsurprisingly,
Walkers method is by far the fastest.

\vskip 0.1in
\noindent {\bf Algorithm L}

\noindent In this algorithm, $1 \le l \le 2n$, and $l$ simply indicates the position
we are trying to set.  $k$ is the value we are trying, and $k = p_j$.
$y_l$ is the $j$ that we chose.  $x_i = 0$ means the value is unset,
and $x_i < 0$ means that value was already used earlier. 

\vskip 0.1in
\noindent {\bf Optimized Algorithm L}

\noindent It's useful to have the fully optimized Algorithm~L in one place, 
incorporating the improvements from exercises (20) and (21).   However, this
is not the Iterator based approach as discussed above.

\noindent This uses auxillary arrays $p_0 p_1 \ldots p_n$, $y_1 \ldots y_{2n}$, 
and $a_1 \ldots a_n$.

\noindent {\bf L1.} [Initialize.] Set $x_1 \ldots x_{2 n} \leftarrow 0 
\ldots 0$, $p_k \leftarrow k + 1$ for $0 \leq k < n$, $p_n \leftarrow 0$, 
$a_{1} \ldots a_{n} \leftarrow 0 \ldots 0$, $l \leftarrow 1$,
$n^{\prime} = n - [n \rm{\, is\, even}].$
\vskip 0.05in
\noindent {\bf L2.} [Enter level $l$.] Set $k \gets p_0$.  If $k = 0$, visit
$x_1 \ldots x_{2n}$, optionally visit $x_{2n} \ldots x_1$, and 
go to {\bf L5}.  Otherwise, set $j \leftarrow 0$, and
while $x_l < 0$, go to {\bf L5} if ($l = \lfloor n / 2 \rfloor$ and 
$a_{n^{\prime}} = 0$) or ($l \ge n - 1$ and $a_{2n - l - 1} = 0$), otherwise
set $l \leftarrow l + 1$.
\vskip 0.05in
\noindent {\bf L3.} [Try $x_l = k$.] (At this point we have $k = p_j$).  
If $l + k + 1 > 2n$ goto {\bf L5}.  If $l = \lfloor n / 2 \rfloor$ and 
$a_{n^{\prime}} = 0$, while $k \ne n^{\prime}$ set $j \leftarrow k$, 
$k \leftarrow p_k$.  If $l \ge n - 1$ and $a_{2n - l - 1} = 0$, while 
$l + k + 1 \ne 2 n $ set $j \leftarrow k$, $k \leftarrow p_k$.  If 
$x_{l + k + 1} = 0$, set $x_l \leftarrow k$, $x_{l + k + 1} \leftarrow - k$,
$a_k \leftarrow 1$, $y_l \leftarrow j$, $p_j \leftarrow p_k$, 
$l \leftarrow l + 1$, and return to {\bf L2}.
\vskip 0.05in
\noindent {\bf L4.} [Try again.] (We've found all solutions that begin with 
$x_1 \ldots x_{l-1} k$ or something smaller.) Set $j \leftarrow k$ and 
$k \leftarrow p_j$, then go to {\bf L3} if $k \ne 0$.
\vskip 0.05in
\noindent {\bf L5.} [Backtrack.] Set $l \leftarrow l - 1$.   If $l = 0$ then 
terminate the algorithm. Otherwise do the following: While $x_l < 0$, set 
$l \leftarrow l - 1$.  Then set $k \leftarrow x_l$, $x_l \leftarrow 0$, 
$x_{l + k + 1} \leftarrow 0$, $a_k \leftarrow 0$, $j \leftarrow y_l$, 
$p_j \leftarrow k$.  If $l = \lfloor n / 2 \rfloor$ and $k = n^{\prime}$
goto {\bf L5}, otherwise goto {\bf L4}.
\vskip 0.1in

\noindent When implemented in c++, the relative timings are 16 $\mu$s and 7.3 $\mu$s 
for $n = 7$ for the raw and optimized Algorithm~L (both visiting both the 
forward and reversed solutions), and 60 ms vs.\ 18 ms for $n = 11$.

\bye

\bye