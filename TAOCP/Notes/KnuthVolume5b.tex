\def\newstep#1{\smallskip \noindent {\bf #1}}
\def\newprob#1{\vskip 0.12in \noindent {\bf #1}}

\topglue 0.5in
\centerline{Notes on Knuth Fascicle 5b: Basic Backtracking}
\vskip 0.3in

\noindent {\bf Timing tests}

I implemented the N-Queens algorithms, including the optimizations
discussed in the problems, in c++ and timed their performance using
Google Benchmark.  The four methods were a  unoptimized
version, the array version of the basic algorithm~B, the
same version modified to use bit twiddling instead of arrays,
and Walkers version.  The execution time runs in the ratio
22:6.3:6.6:1.  So bit twiddling is not helpful, but, unsurprisingly,
Walkers method is by far the fastest.

\vskip 0.1in
\noindent {\bf Optimized Algorithm L}

It's useful to have the fully optimized Algorithm~L in one place, incorporating
the improvements from exercises (20) and (21).

\noindent This uses auxillary arrays $p_0 p_1 \ldots p_n$, $y_1 \ldots y_{2n}$, and
$a_1 \ldots a_n$.

\noindent {\bf L1.} [Initialize.] Set $x_1 \ldots x_{2 n} \leftarrow 0 \ldots 0$,
$p_k \leftarrow k + 1$ for $0 \leq k < n$, $p_n \leftarrow 0$, 
$a_{1} \ldots a_{n} \leftarrow 0 \ldots 0$, $l \leftarrow 1$,
$n^{\prime} = n - [n \rm{\, is\, even}].$
\vskip 0.05in
\noindent {\bf L2.} [Enter level $l$.] Set $k \gets p_0$.  If $k = 0$, visit
$x_1 \ldots x_{2n}$, optionally visit $x_{2n} \ldots x_1$, and 
go to L5.  Otherwise, set $j \leftarrow 0$, and
while $x_l < 0$, go to L5 if ($l = \lfloor n / 2 \rfloor$ and $a_{n^{\prime}} = 0$) 
or ($l \ge n - 1$ and $a_{2n - l - 1} = 0$), otherwise
set $l \leftarrow l + 1$.
\vskip 0.05in
\noindent {\bf L3.} [Try $x_l = k$.] (At this point we have $k = p_j$).  If $l + k + 1 > 2n$
goto L5.  If $l = \lfloor n / 2 \rfloor$ and $a_{n^{\prime}} = 0$, while $k \ne n^{\prime}$
set $j \leftarrow k$, $k \leftarrow p_k$.  If $l \ge n - 1$ and $a_{2n - l - 1} = 0$, 
while $l + k + 1 \ne 2 n $ set $j \leftarrow k$, $k \leftarrow p_k$.  If $x_{l + k + 1} = 0$, 
set $x_l \leftarrow k$, $x_{l + k + 1} \leftarrow -k$,
$a_k \leftarrow 1$, $y_l \leftarrow j$, $p_j \leftarrow p_k$, $l \leftarrow l + 1$, and return to L2.
\vskip 0.05in
\noindent {\bf L4.} [Try again.] (We've found all solutions that begin with $x_1 \ldots x_{l-1} k$
or something smaller.) Set $j \leftarrow k$ and $k \leftarrow p_j$, then go to L3 if $k \ne 0$.
\vskip 0.05in
\noindent {\bf L5.} [Backtrack.] Set $l \leftarrow l - 1$.   If $l = 0$ then terminate the algorithm.
Otherwise do the following: While $x_l < 0$, set $l \leftarrow l - 1$.  Then set 
$k \leftarrow x_l$, $x_l \leftarrow 0$, $x_{l + k + 1} \leftarrow 0$, $a_k \leftarrow 0$, 
$j \leftarrow y_l$, $p_j \leftarrow k$.  If $l = \lfloor n / 2 \rfloor$ and $k = n^{\prime}$
goto L5, otherwise goto L4.

\bye