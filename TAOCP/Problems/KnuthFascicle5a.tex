\def\newstep#1{\smallskip \noindent {\bf #1}}
\def\newprob#1{\vskip 0.12in \noindent {\bf #1}}

\topglue 0.5in
\centerline{\tt Knuth Fascicle 5a}
\vskip 0.3in

\noindent {\bf Problem 1} {\it Non-transitive dice}\hfil\break
(a) ${\rm Pr}\left(A > B\right) = 2 / 3$ (chance of $A$ rolling a 5) $\times 5 / 6$
(chance that a 5 beats $B$) = 5 / 9.  And ${\rm Pr}\left(B > C\right) = 2 / 6 \times 2 / 6$
($B = 3$) $ + 3 / 6 \times 4 / 6$ ($B = 4$) $+ 1 / 6 \times 4 / 6$ ($B = 6$)
$ = 1 / 9 + 1 / 3 + 1 / 9 = 5 / 9$, ${\rm Pr}\left(C > A\right) = 2 / 6 \times 2 / 6 +
2 / 6 \times 2 / 6 + 2 / 6 = 1 / 9 + 1 / 9 + 1 / 3 = 5 / 9$.\hfil\break

\newprob{Problem 6} {\it Pairwise independence does not imply $k$-wise independence}\hfil\break
Note that these can't be independent, or there would be no way to have a vector
with two set values have non-zero probability but 1 or 3 have zero probability.
Also note that there are $n \choose 2$ ways to have $x_1 + \ldots + x_n = 2$,
so the chance that 2 are set is ${n \choose 2} \times {1 \over \left(n - 1\right)^2} =
{n \left(n - 1\right) \over 2} \times {1 \over \left(n - 1\right)^2} = {n \over 2 \left(n - 1\right)}$.
Combined with the chance that zero are set ($\left(n - 2\right) \over \left(2 n - 2\right)$),
the total probability is one.

In any case, we want to compute the probability that $X_i = 1$.  Given $X_i = 1$, there are $n - 1$
ways to set $X_j = 1$ for $i \neq j$, all of equal probability, and all other settings have zero probability.
Therefore, the probability that $X_i = 1$ must be $1 / \left(n - 1\right)$ so that the sum of
all those arrangements adds up to the right value.  What about $\left(X_i, X_j\right) = \left(0, 1\right)$ 
for $i \neq j$? Well, there are $n-2$ other choices for $X_k = 1$, so the probability
must be $n - 2 \over \left(n - 1\right)^2$, and we must have the same probability
for $\left(X_i, X_j\right) = \left(1, 0\right)$.  We already know that the chances of
$\left(X_i, X_j\right) = \left(1, 1\right) = {1 \over \left(n - 1\right)^2}$ for $i \neq j$,
so by subtraction we must have the chances of them both being 0 of
$\left(n - 2\right)^2 \over \left(n - 1\right)^2$.

Now note that these can be written as the probability that ${\rm Pr} \left(X_i, X_j\right) = 
\left(0, 0\right), \left(0, 1\right), \left(1, 0\right), \left(1, 1\right) =
p_0^2, p_0 p_1, p_1 p_0, p_1^2$ for $p_0 = {\left( n - 2\right) \over \left(n - 1\right)}$,
$p_1 = {1 \over n - 1}$, which constitutes 2-wise independence.
But we can't have $\left(X_i, X_j, X_k\right) = \left(0, 0, 0\right)$ for any distinct
$i, j, k$, and the same is true for any combination of 4 or more.  So they are 2-wise
independent, but no more.

\newprob{Problem 46} {\it Explain why ${\rm E}\left(X^2 | X > 0\right) \geq 
\left({\rm E} \left(X | X > 0\right)\right)^2$}\hfil\break
There are (at least) two ways to look at this, and neither of them depends on the $X > 0$
part.  So I'll drop that.  First, there's the usual ${\rm E}\left( \left(X - {\rm E} \, X\right)^2 \right) =
{\rm E} \, X^2 - \left({\rm E}\, X\right)^2$ where the square is manifestly non-negative.
Second, $X^2$ is convex so we can simply use Jensen's inequality.
\bye