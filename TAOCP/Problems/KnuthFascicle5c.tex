\def\newstep#1{\smallskip \noindent {\bf #1}}
\def\newprob#1{\vskip 0.12in \noindent {\bf #1}}

\topglue 0.5in
\centerline{\tt Knuth Fascicle 5c}
\vskip 0.3in

\noindent {\bf Problem 3} {\it Solving exact cover problems using
linear algebra}\hfil\break
(a) Together $x_2 + x_4 = 1$ and $x_2 + x_4 + x_6 = 1$
imply that $x_6 = 0$.  Combined that with $x_1 + x_6 = 1$
gives $x_1 = 1$.  In turn with $x_1 + x_3 = 1$ this implies
$x_3 = 0$, then with $x_3 + x_5 = 1$ we have $x_5 = 1$.
Then $x_3 + x_4 = 1 \rightarrow x_4 = 1$ and finally
$x_2 + x_4 = 1 \rightarrow x_2 = 0$.  Altogether
$\vec{x} = [1, 0, 0, 1, 1, 0]$.\hfil\break
(b) Because in general there will be many, many more
rows than columns (e.g., $m \gg n$), and so we the solution
will not be uniquely determined.

\newprob {Problem~4} {\it Exact cover on graph.}\hfil\break
You are chosing edges such that each vertex has exactly one incident edge.
This has various interpretations depending on the graph.  If it is bipartite, for
example, each solution choses the vertices of one part.

\newprob {Problem~6} {\it Size of data structures in Algorithm~D}\hfil\break
The items occupy $N+1$ locaions of size 2 + the space to represent the names.
Then we have $N$ header nodes plus $M+1$ spacers plus the $L$ options,
each taking 3 elements, so $3\left(L + N + M + 1\right)$ locations.

\newprob {Problem~7} {\it Explain specific values in Table~1}\hfil\break
(a) {\tt TOP(23) = -4} because it is a spacer following option 4 (this is not
really part of the algorithm; all that matters is that it is negative.\hfil\break
(b) {\tt DLINK(23) = 25} because that is the last element of the next option
(option 5).

\newprob {Problem~18} {\it MRV Heuristic}\hfil\break
Simply walk the list of active items and keep track of the minimum encountered
so far.\hfil\break
{\bf M1.} [Initialize] If $RLINK\left(0\right) = 0$, terminate unsuccessfully (there are no
 active items).  Otherwise, set $m \leftarrow \infty$, $p \leftarrow 0$ ($m$ will hold
 the current minimum).\hfil\break
{\bf M2.} [Walk forward] Set $p \leftarrow RLINK\left(p\right)$.\hfil\break
{\bf M3.} [Done?] If $p = 0$ goto {\bf M6}.\hfil\break
{\bf M4.} [Compare] Set $l \leftarrow LEN\left(p\right)$.  If $l < m$ set $m \leftarrow l$,
 $i \leftarrow p$.\hfil\break
{\bf M5.} [Early terminate] If $m \ne 0$, goto {\bf M2}.\hfil\break
{\bf M6.} [Exit] Return $i$.\hfil\break
\bye