\topglue 0.5in
\centerline{\tt Knuth 7.2.2.2: Satisfiability}
\vskip 0.3in

\noindent {\bf Problem 1} The shortest unsatisfiable set of clauses is the
empty clause $\epsilon$, the shortest satisfiable set is the empty set $\emptyset$.

\vskip 0.08in \noindent {\bf Problem 121}: Lower level operations in Algorithm~A\hfil\break
{\bf A3} [Remove $\bar l$].  Because of the reverse order the variables are
stored in, and the fact that the algorithm chooses values for $x_1, x_2, \ldots, x_n$
in order, we can remove $\bar l$ from all clauses that contain it simply by
reducing the value of {\tt SIZE} for those clauses by one.  We can find the clauses
by following the circular links until we get back to the head of the list.
The messy bit is that when a clause would go to size 0 we need to undo
what we have already done because the set of clauses has become unsatisfiable
given the current assignments.  Thus: $p \gets {\tt F}\left(l \oplus 1\right)$ 
(follow the forward links to clauses that contain $\bar l$).  Repeat until
we come back to the head of the list (so, as long as $p \ge 2 n + 2$):
i) Get the clause number: $c \gets {\tt C}\left(p\right)$. ii) If ${\tt SIZE}\left(c\right) > 1$,
 decrease it by 1 and $p \gets {\tt F}\left(p\right)$.  If not, we need to undo what
 we've already done.  So interrupt the previous
 loop, set $p = {\tt B}\left(p\right)$ and repeat as long as $p \ge 2 n + 2$:
 i) Get the clause $c \gets {\tt C}\left(p\right)$. ii) ${\tt SIZE}\left(c\right) += 1$.
 iii) $p \gets {\tt B}\left(p\right)$ and go to A5.\hfil\break
{\bf A4} [Remove all clauses that contain $l$].  Here we simply iterate
over all other literals in each clause that contains $l$ and unlink them from
the circular lists.  So: set $p \gets {\tt F}\left(l\right)$ and repeat as
long as $p \ge 2 n + 2$: i) set $c \gets {\tt C}\left(p\right)$ ($c$ is
the clause number), ii) for ${\tt START}\left(c\right) \le k < {\tt START}\left(c\right)
 + {\tt SIZE}\left(c\right) - 1$ (loop over all other literals in this clause)
 remove the literal $k$ using the procedure following below, iii) set $p \gets {\tt F}\left(p\right)$.
 To remove the literal pointed to by $k$ a) set $d \gets {\tt F}\left(k\right)$,
 $e \gets {\tt B}\left(k\right)$, ${\tt B}\left(d\right) \gets e$, ${\tt F}\left(e\right) \gets d$
 (the variable is removed from the circular list), b) decrement the counter for that
 literal: ${\tt C}\left({\tt L}\left(k\right)\right) -= 1$. The final overall step is
 to set $a \gets a - {\tt C}\left(l\right)$ (correct the number of active clauses),
 and $d \gets d + 1$ (we've set another variable).\hfil\break
 {\bf A7} [Reactivate $l$s clauses] This takes advantage of the Dancing Links
 procedure.  Because we haven't changed {\tt F} or {\tt B} for the variables
 we removed in A4, we can put them back in efficiently.  So, $p \gets {\tt B}\left(l\right)$
 and while $p \ge 2 n + 2$ do: $c \gets {\tt C}\left(p\right)$, and for
 ${\tt START}\left(c\right) \le k < {\tt START}\left(c\right) + {\tt SIZE}\left(c\right) - 1$
add the literal $k$ by doing $d \gets {\tt F}\left(k\right)$, $e \gets {\tt B}\left(k\right)$,
${\tt B}\left(d\right) \gets k$, ${\tt F}\left(e\right) \gets k$, and 
${\tt C}\left({\tt L}\left(k\right)\right) += 1$, and $p \gets {\tt B}\left(p\right)$.
Then, finally, $a \gets a + {\tt C}\left(l\right)$.\hfil\break
{\bf A8} [Unremove $\bar l$] $p \gets {\tt F}\left(l \oplus 1\right)$,
and repeat until $p \ge 2 n + 2$: $c \gets {\tt C}\left(p\right)$,
${\tt SIZE}\left(c\right) += 1$, $p \gets {\tt F}\left(p\right)$.  Then go to A5.

\bye