\def\newstep#1{\smallskip \noindent {\bf #1}}
\def\newprob#1{\vskip 0.12in \noindent {\bf #1}}

\topglue 0.5in
\centerline{\tt Knuth 7.2.2.2: Satisfiability}
\vskip 0.3in

\newprob {Problem 1} The shortest unsatisfiable set of clauses is the
empty clause $\epsilon$, the shortest satisfiable set is the empty set $\emptyset$.

\newprob {Problem 121}: Lower level operations in Algorithm~A
\newstep {A3} [Remove $\bar l$].  Because of the reverse order the variables are
stored in, and the fact that the algorithm chooses values for $x_1, x_2, \ldots, x_n$
in order, we can remove $\bar l$ from all clauses that contain it simply by
reducing the value of {\tt SIZE} for those clauses by one.  We can find the clauses
by following the circular links until we get back to the head of the list.
The messy bit is that when a clause would go to size 0 we need to undo
what we have already done because the set of clauses has become unsatisfiable
given the current assignments.  Thus: $p \gets {\tt F}\left(l \oplus 1\right)$ 
(follow the forward links to clauses that contain $\bar l$).  Repeat until
we come back to the head of the list (so, as long as $p \ge 2 n + 2$):
i) Get the clause number: $c \gets {\tt C}\left(p\right)$. ii) If ${\tt SIZE}\left(c\right) > 1$,
 decrease it by 1 and $p \gets {\tt F}\left(p\right)$.  If not, we need to undo what
 we've already done.  So interrupt the previous
 loop, set $p = {\tt B}\left(p\right)$ and repeat as long as $p \ge 2 n + 2$:
 i) Get the clause $c \gets {\tt C}\left(p\right)$. ii) ${\tt SIZE}\left(c\right) += 1$.
 iii) $p \gets {\tt B}\left(p\right)$ and go to A5.
\newstep {A4} [Remove all clauses that contain $l$].  Here we simply iterate
over all other literals in each clause that contains $l$ and unlink them from
the circular lists.  So: set $p \gets {\tt F}\left(l\right)$ and repeat as
long as $p \ge 2 n + 2$: i) set $c \gets {\tt C}\left(p\right)$ ($c$ is
the clause number), ii) for ${\tt START}\left(c\right) \le k < {\tt START}\left(c\right)
 + {\tt SIZE}\left(c\right) - 1$ (loop over all other literals in this clause)
 remove the literal $k$ using the procedure following below, iii) set $p \gets {\tt F}\left(p\right)$.
 To remove the literal pointed to by $k$ a) set $d \gets {\tt F}\left(k\right)$,
 $e \gets {\tt B}\left(k\right)$, ${\tt B}\left(d\right) \gets e$, ${\tt F}\left(e\right) \gets d$
 (the variable is removed from the circular list), b) decrement the counter for that
 literal: ${\tt C}\left({\tt L}\left(k\right)\right) -= 1$. The final overall step is
 to set $a \gets a - {\tt C}\left(l\right)$ (correct the number of active clauses),
 and $d \gets d + 1$ (we've set another variable).
\newstep {A7} [Reactivate $l$s clauses] This takes advantage of the 
 Dancing Links procedure.  Because we haven't changed {\tt F} or {\tt B} for the variables
 we removed in A4, we can put them back in efficiently.  So, $p \gets {\tt B}\left(l\right)$
 and while $p \ge 2 n + 2$ do: $c \gets {\tt C}\left(p\right)$, and for
 ${\tt START}\left(c\right) \le k < {\tt START}\left(c\right) + {\tt SIZE}\left(c\right) - 1$
add the literal $k$ by doing $d \gets {\tt F}\left(k\right)$, $e \gets {\tt B}\left(k\right)$,
${\tt B}\left(d\right) \gets k$, ${\tt F}\left(e\right) \gets k$, and 
${\tt C}\left({\tt L}\left(k\right)\right) += 1$, and $p \gets {\tt B}\left(p\right)$.
Then, finally, $a \gets a + {\tt C}\left(l\right)$.
\newstep {A8} [Unremove $\bar l$] $p \gets {\tt F}\left(l \oplus 1\right)$,
and repeat until $p \ge 2 n + 2$: $c \gets {\tt C}\left(p\right)$,
${\tt SIZE}\left(c\right) += 1$, $p \gets {\tt F}\left(p\right)$.  Then go to A5.

\vskip 0.08in \noindent {\bf Problem 109} Find the lexicographically smallest
solution.\hfil\break
First, find a solution.  If you can't, terminate unsuccessfully.  Then take your
solution and find the first 1.  Say it's at position $j$.  Next, try to find a solution
with $x_1 x_2 \ldots x_j = 0$ by adding additional clauses that force those
to your problem.  If you do find a solution, repeat.  If not, say that then next
1 after $j$ is in position $k$.  Now try for a solution with $x_1 \ldots x_{j-1} = 0$
and $x_k = 0$, repeating until there are no more $k$s to try.

Finding the largest is similar, but somewhat more complicated.

\newprob {Problem 124} Fill in the steps in B2.\hfil\break
We want to walk the linked list of clauses that watch $\bar l$.  So, set
$c \gets W_{\bar l}$.  If $c = 0$, there are no clauses watches $\bar l$, so
terminate successfully.  Otherwise, while $c \ne 0$ do: $i_{min} \gets {\tt START}
\left(c\right)$, $i_{max} \gets {\tt START}\left(c - 1\right)$, $c^{\prime} \gets 
{\tt LINK}\left(c\right)$ (this will be the update for the next clause).
Now, for $i_{min} < k < i_{max}$, set $l^{\prime} \gets {\tt L}\left(k\right)$.
Now, if the literal represented by $l^{\prime}$ isn't false, we've found
something else to watch, so we swap it to the front of the clause (meaning we
watch it): ${\tt L}\left(i_{min}\right) \gets l^{\prime}$, ${\tt L}\left(k\right) \gets \bar l$,
${\tt LINK}\left(c\right) \gets W_{l^{\prime}}$, $W_{l^{\prime}} \gets c$, $c \gets c^{\prime}$
and exit the $k$ loop (we now move on to the next clause watching $\bar l$.
If we reach the end of the $k$ loop without finding such a literal, then the formula
is unsatisfiable with the current settings, so we go to step B5.

Now, how do we check if a literal is already false?  Well, Algorithm~B steps
through the literals in order, so the condition for the literal to not be false is
$| l^{\prime} | > d$ (not set) or $| l^{\prime} | + m_{| l^{\prime} |}$ is even (is set,
but set to be true.

\newprob {Problem 125} Modify Algorithm~B to find all solutions.
\newstep {B2}  [Visit or choose] If $d > n$, visit the solution represnted by $m_1 \ldots m_n$
(recall $x_j \gets 1 \oplus \left( m_j \& 1 \right)$ constructs $x$) and go to B6.
\newstep {B6} [Backtrack or terminate] Terminate if $d=1$.  Otherwise set $d \gets d - 1$.

\newprob {Problem 126} Find the next step of Algorithm~D on (59)\hfil\break
The book has a mistake -- the active ring is $\left(6\,9\,7\,8\right)$, since 
$\bar 5$ was just set.  So, first it looks at 6, but there are no unit clauses forced
(the possibilities would be $\bar 1 \bar 2 \bar 6$, $\bar 1 \bar 5 \bar 6$, $\bar 2 \bar 5 \bar 6$,
$346$, none of which are present). So next it tries 9, which {\it does} exist in a unit 
clause (1 5 9), so this is chosen.  That affects the watch list of $\bar 9 \bar 3 \bar 6$ --
$39$ are set, so we switch to watching 6 by converting it to $\bar 6 \bar 3 \bar 9$.
The active ring loses 9, so becomes $\left(6\,7\,8\right)$.

\newprob {Problem 127} Move codes for (59) right before and after backtrack.\hfil\break
$m = 1 1 4 1 4 5 4 5$; note that we don't need $h$, we can just follow the steps
in order.  But $h = 1 2 3 4 6 9 7 8$.  After the backtrack, $m = 1 1 4 2$ (we try
a different setting for 4 after the first one failed.

\newprob {Problem 128} Show what Algorithm~D does on the Rivest clauses (6).
\hfil\break
Following the layout used in problem~123, we have
$$
\matrix{
p =       & 0 & 1 & 2 & 3 & 4 & 5 & 6 & 7 & 8 & 9 & 10 & 11 & 12 & 13 & 14 
   & 15 & 16 & 17 & 18 & 19 & 20 & 21 & 22 & 23 \cr
{\tt L} = & 5 & 2 & 9 & 3 & 9 & 7 & 8 & 7 & 5 & 6 & 5 & 3 & 4 & 3 & 8
 &2 & 8 & 6 & 9 & 6 & 4 & 7 & 4 & 2 \cr
}
$$
with ${\tt START}\left(j\right) = 24 - 3 j$,
$W_l = \{3, 7, 4, 8, 5, 1, 6, 2\}$ for $2 \le l < 10$, and
${\tt LINK}\left(j\right) = 0$ for $1 \le j \le 8$.
\newstep {D1} sets ${\tt NEXT} = \{2, 3, 4, 1\}$ with $h = 1$, $t = 4$.
 The active ring is $\left(1 2 3 4\right)$.
\newstep {D2} sets $k \gets 4$.
\newstep {D3} There are no unit clauses yet.
\newstep {D4} $h \gets {\tt NEXT}\left(t\right) = {\tt NEXT}\left(4\right) = 1$,
 $m_1 \gets 1$, so we are trying $\bar 1$.
\newstep {D5} $d \gets 1$, $h_1 \gets 1$, $k \gets 1$.  $t \ne k$, so delete $1$
from the ring by setting ${\tt NEXT}\left(4\right) \gets {\tt NEXT}\left(1\right) = 2$
and $h \gets 2$.  We now have ${\tt NEXT} = \{2 3 4 2\}$, and the active
ring is $\left(2 3 4 \right)$.
\newstep {D6} We update the watches by doing $b \gets \left(m_1 + 1\right)\, {\rm mod}
\, 2 = 0$, $x_1 \gets 0$, and clearing the watch list for $1$, which is watched by $C_3 = 
341$.  The net effect of the steps carried out in exercise~130 is to change
this clause to $3 1 4$ by changing the relevant part of the link table ($15 \le p \le 18$)
from $2 8 6$ to $8 2 6$, setting $W = \{0, 7, 4, 8, 5, 1, 3, 2 \}$ and ${\tt LINK}\left(3\right)
= 6$.
\newstep {D2} $k \gets 4$.
\newstep {D3} $h \gets {\tt NEXT}\left(k\right) = 2$.  There are no unit clauses.
\newstep {D4} $h \gets {\tt NEXT}\left(t\right) = 2$, $m_2 \gets 1$ (trying $\bar 2$).
\newstep {D5} $d \gets 2$, $h_2 \gets 2$, $k \gets 2$.  We have $x = \{ 0 0 - - \}$
and $h = \{1 2 - - \}$.  We delete $2$ from the ring by setting 
$h \gets 3$ and ${\tt NEXT} = 2 3 4 3$ -- the active ring is now $\left(3 4\right)$.
\newstep {D6} Clear the watch list for 2, which is $C_4 = 4 \bar 1 2$.
This is accomplished by changing $C_4$ to $4 2 \bar 1$ by
changing the values of the link table $12 \le p \le 14$ to $348$,
changing $W = \{0, 4, 0, 8, 5, 1, 3, 2 \}$ and setting ${\tt LINK}\left(4\right) = 7$.
\newstep {D2} $k \gets 4$.
\newstep {D3} Now we do have a unit clause: $C_1$ has become $\bar 3$
since we have $\bar 1 \bar 2$.  $h \gets 3$, so $f \gets 2$, so
$m_3 \gets 5$ (forced $\bar 3$), $t \gets 4$.
\newstep {D5} $d \gets 3$, $h_3 \gets 3$, $k \gets 3$ and
 $h \gets 4$, ${\tt NEXT} = 2 3 4 4$ (the active ring is 4).
\newstep {D6} $x_3 \gets b \gets 0$, and clear the watch list for 3, which is
$C_5 = \bar 1 \bar 2 3$, and which is changed to $\bar 1 3 \bar 2$.
At this point, we have
$$
\matrix{
p =       & 0 & 1 & 2 & 3 & 4 & 5 & 6 & 7 & 8 & 9 & 10 & 11 & 12 & 13 & 14 
   & 15 & 16 & 17 & 18 & 19 & 20 & 21 & 22 & 23 \cr
{\tt L} = & 5 & 2 & 9 & 3 & 9 & 7 & 8 & 7 & 5 & 5 & 6 & 3 & 3 & 4 & 8
 &8 & 2 & 6 & 9 & 6 & 4 & 7 & 4 & 2 \cr
}
$$
with $W = \{0, 4, 0, 5, 0, 1, 3, 2 \}$ and ${\tt LINK} = \{0, 0, 6, 7, 8, 0, 0, 0\}$,
$m = \{1 1 5 -\}$, $h = \{1 2 3 -\}$, and ${\tt NEXT} = 2 3 4 4$.
\newstep {D2} $k \gets 4$.
\newstep {D3} Now we have two unit clauses, $4, \bar 4$ from $ 341 , 234$ respectively.
So we have to backtrack!
\newstep {D7} $t \gets 4$.  $m_3 = 5 \ge 2$, so $k \gets h_3 = 3$, $x_3$ is unset,
$W_7 \ne 0$ so $h \gets 3$, ${\tt NEXT} = 2 3 4 3$ () and $d \gets 2$.  The
active ring is now $\left(3 4\right)$. ($m_2 < 2$).
\newstep {D8} $m_2 \gets 3 - 1 = 2$ (trying $x_2 = 1$ after $x_2 = 0$ failed).

This is enough to see how the algorithm works.  In particular, note that the
clause re-ordering done by exercise~130 is different than the solution Knuth
gives for this problem -- which is fine, if perhaps slightly confusing.

Anyways, to spell it out:
$$
\matrix{
{\rm Ring}     & x_1 & x_2 & x_3 & x_4 & {\rm Units} & {\rm Choice} & {\rm Changed} & 
 {\rm Notes} \cr
\left(1\,2\,3\,4\right) & - & - & - & - & & \bar 1 & 341 \rightarrow 314 & w_3=7, m_1 = 1\cr
\left(2\,3\,4\right) & 0 & - & - & - & & \bar 2 & 4 \bar 1 2 \rightarrow 4 2 \bar 1 &
 w_5 = 8, m_2 = 1\cr
\left(3\, 4\right) & 0 & 0 & - & - & \bar 3 & \bar 3 & \bar 1 \bar 2 3 \rightarrow \bar 1 3 \bar 2 &
h = 1\,2\,3\,- , m = 1\,\,1\,5\,- \cr
\left(4\right) & 0 &  0 & 0 & - & 4, \bar 4 & {\rm Backtrack} & & 
{\rm State} = \{1 2 \bar3, 2 3 \bar 4, 3 1 4, 4 2 \bar1, \bar1 3 \bar 2, \bar 2 \bar 3 4,
 \bar 3 \bar 4 \bar 1, \bar 4 1 \bar 2 \}\cr
\left( 3\, 4 \right) & 0 & - & - & - & & 2 & \bar 1 3 \bar 2 \rightarrow \bar 1 \bar 2 3
& m = 1\,2\,-\,- \cr
& & & & & & & \bar 4 1 \bar 2 \rightarrow \bar 2 1 \bar 4 & \cr
\left(3\,4\right) & 0 & 1 & - & - & \bar 4 & \bar 4 & \bar2 \bar 3 4 \rightarrow \bar 2 4 \bar 3 &
 h = 1\,2\,4\,-, m = 1\,2\,5\,- \cr
\left(3\right) & 0 & 1 & - & 0 & \bar 3, 3 & {\rm Backtrack} & & 
{\rm State} = \{ 1 2 \bar 3, 2 3 \bar 4, 4 1 3, 4 2 \bar 1, \bar 1 \bar 2 3, \bar 2 4 \bar 3,
\bar 3 \bar 4 \bar 1, \bar 2 1 \bar 4 \} \cr
\left(4 3 \right) & - & - & - & - & & 1 & \bar 3 \bar 4 \bar 1 \rightarrow \bar 3 \bar 1 \bar 4 &
 m = 2\,-\,-\,- \cr
 & & & & & & & 4 2 \bar 1 \rightarrow 4 \bar 1 2 & \cr
\left( 2 \, 4 \, 3\right) & 1 & - & - & - & & 2 & {\rm None} & m = 2\, 0\, - \, - \cr
\left(4\, 3\right) & 1 & 1 & - & - & 3 & 3 & 1 2 \bar 3 \rightarrow 1 \bar 3 2 &
 m = 2\,0\,4\,-\cr
 & & & & & & & \bar 2 4 \bar 3 \rightarrow \bar 2 \bar 3 4 & \cr
\left(4 \right) & 1 & 1 & 1 & - & 4, \bar 4 & {\rm Backtrack} & &
 {\rm State} = \{ 1 \bar 3 2, 2 3 \bar 4, 4 1 3, 4 \bar 1 2, \bar 1 \bar 2 3, \bar 2 \bar 3 4,
 \bar 3 \bar 1 \bar 4, \bar 2 1 \bar 4 \} \cr
\left(3\, 4\right) & 1 & - & - & - & & \bar 2 & 1 \bar 3 2 \rightarrow 1 2 \bar 3 & \cr
 & & & & & & & 4 \bar 1 2 \rightarrow 2 \bar 1 4 & \cr
\left(3 \, 4 \right) & 1 & 0 & - & - & 4 & 4 & \bar 2 1 \bar 4 \rightarrow \bar 2 \bar 4 1 & \cr
 & & & & & & & 2 3 \bar 4 \rightarrow 2 \bar 4 3 & \cr
 & & & & & & & \bar 3 \bar 1 \bar 4 \rightarrow \bar 4 \bar 1 \bar 3 & \cr
\left(3\right) & 1 & 0 & - & 1 & 3, \bar 3 & {\rm Failure} & \cr
}
$$

\newprob {Problem~129} Subroutine to return 1 if a literal $l$ is a unit.
\newstep {K1} [Initialize] Set $c \gets W_l$.
\newstep {K2} [Terminate?] If $c = 0$, return 0 (no unit clauses found).
\newstep {K3} [Start a new clause] $s \gets {\tt START}\left(c\right) + 1$.
\newstep {K4} [Return success if all other literals false] 
  If $s = {\tt START}\left(c - 1\right)$, return 1 (a unit clause was found).
\newstep {K5} [Move to next literal? ]
  If this literal is false ($x_{\left| L \left(s\right) \right|} = {\tt L} \left(s\right) \& \, 1$,
recalling that $\left| {\tt L}\left(s\right) \right| = {\tt L}\left(s\right) \gg 1$)
 set $s \gets s + 1$ and return to {\bf K4}.
\newstep {K6} [Try next clause] Set $c \gets {\tt LINK}\left(c\right)$ and go to {\bf K2}.


\newprob {Problem~249} Carry out Algorithm~IA on $\{1234, 1 \bar 2, \bar 1 \bar 2 \bar 3,
\bar 1 3, 2 \bar 3, 3 \bar 4 \}$\hfil\break
Starting from the beginning:
\newstep {I1} $d \gets 0$, $m \gets 6$
\newstep {I2} $d \gets 1$, $l_1 \gets 3$, leaving $\{-, 1 \bar 2, \bar 1 \bar 2, -, 2, - \}$
\newstep {I2} $d \gets 2$, since the shortest is $C_5 = 2$.  So $l = \left(3\,2\right)$,
 and we have $\{-, 1, \bar 1, -, -, - \}$.
\newstep {I2} $d \gets 3$, $l_3 \gets 1$, $l = \left(3\,2\,1\right)$.
\newstep {I3} Now $C_3 = \bar 1 \bar 2 \bar 3$ is falsified (so $i \gets 3$).
\newstep {I4} Set $l_3 \gets \bar 1$, and now $C_2 = 1 \bar 2$ is falsified.  So $j = 2$.
\newstep {I5} $m \gets 7$, learn $C_7 = C_2 \diamond C_3 = 1 \bar 2 \diamond \bar 1 \bar 2 \bar 3 = 
 \bar 2 \bar 3$.  Now $d \gets 2$ ($l_2 = \bar 2$, and 2 is in the new clause $\bar 2 \bar 3$.
 Also, $i \gets 7$.
\newstep {I4} Now we try to learn things from our new clause. $l_2 \gets \bar 2$ (so
$l \left(3\, \bar 2\right)$), and now $C_5 = 2 \bar 3$ is falsified (so $j \gets 5$).
\newstep {I5} Learn $C_8 = C_5 \diamond C_7 = 2 \bar 3 \diamond \bar 2 \bar 3 = \bar 3$.
  Now $d \gets 1$ ($\bar {l_1}$ is the only literal in the new clause), $l_1 \gets \bar 3$, $i \gets 8$.
\newstep {I4} No clauses are falsified.
\newstep {I2} $d \gets 2$, and the unsatisfied clauses are $\{124, 1 \bar 2, -, \bar 1, -, \bar 4, -, -\}$.
  We can chose either $\bar 2$ or $\bar 4$.  Choosing the former (which now diverges from Knuth's
  solution), $l_2 \gets \bar 2$.
\newstep {I3} No clauses are falsified.
\newstep {I2} $d \gets 3$, $l_3 \gets \bar 2$, leaving clauses $4, \bar 4$.
\newstep {I3} None falsified.
\newstep {I2} $d \gets 4$, $l_4 \gets 4$, so we have $l = \left(\bar 3\, \bar 1\, \bar 2\, 4\right)$.
\newstep {I3} $C_6 = 3 \bar 4$ is falsified, so $i \gets 6$
\newstep {I4} Trying $l_4 \gets \bar 4$, and $C_1 = 1234$ is falsified ($j \gets 1$).
\newstep {I5} Learn $C_9 = 1234 \diamond 3 \bar 4 = 123$.  Now $d \gets 3$, $i \gets 9$.
\newstep {I4} $l_3 \gets 2$, so $l = \left(\bar 3\, \bar 1\, 2\right)$.  This falsifies $C_2 = 1 \bar 2$,
 $j \gets 2$.
\newstep {I5} Learn $C_10 = 123 \diamond 1 \bar 2 = 13$, $i \gets 10$, $d \gets 2$.
\newstep {I4} $l_2 \gets 1$, $l = \left(\bar 3\,1\right)$, and $C_4 = \bar 1 3$ is falsified.
\newstep {I5} Learn $C_11 = C_{10} \diamond C_4 = 1 \bar 2 \diamond \bar 1 3 = 3$,
 and $d \gets 1$, $i \gets 11$.
\newstep {I4} $l_1 \gets 3$, $C_8 = \bar 3$ is falsified.
\newstep {I5} And finally we learn $C_12 = C_11 \diamond C_8 = \bar 3 \diamond 3 = \epsilon$,
 so we have shown the original clauses are unsatisfiable.  The final set of clauses is
 $\{1234, 1 \bar 2, \bar 1 \bar 2 \bar 3,
\bar 1 3, 2 \bar 3, 3 \bar 4, \bar 2 \bar 3, \bar 3, 1 2 3, 1 3, 3, \epsilon \}$.

\bye