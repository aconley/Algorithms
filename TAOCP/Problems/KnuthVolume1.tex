\topglue 0.5in
\centerline{\tt Knuth 1.1}
\vskip 0.5in

\noindent
{\bf Problem 1} $t \leftarrow a, a \leftarrow b, b \leftarrow c, c \leftarrow d, d \leftarrow t$

\vskip 0.1in
\noindent
{\bf Problem 4} $m \leftarrow 2166$ and $n \leftarrow 6099$.  After the first division, $r$ is 
2166, so we set $m \leftarrow 6099$, $n \leftarrow 2166$.  Next, $r$ is 1767, so 
$m \leftarrow 2166$, $n \leftarrow 1767$.  Then $m \leftarrow 1767$, $n \leftarrow 399$,
followed by $m \leftarrow 399$, $n \leftarrow 171$, and $m \leftarrow 171$, $n\leftarrow 57$.
Finally, 57 divides evenly into 171, so the GCD is 57.

\vskip 0.1in
\noindent
{\bf Problem 6} For all but a vanishing fraction of $n$ values, $n > m$.  So the first step
will simply swap $m$ and $n$, leaving us with the $T_m$ problem.  Therefore, $U_m = T_m+1$.

\vskip 0.5in
\centerline{\tt Knuth 1.2.1}
\vskip 0.5in

\noindent
{\bf Problem 2} The $n=2$ step is skipped.  Trying to apply the formula to that case
 requires the assumption that $a^{-1} = 1$, which is not true in general.

\vskip 0.1in
\noindent
{\bf Problem 4}  Start with $n=1$ and $n=2$.  We have $F_1 = 1 > \phi^{-1}$, 
 and $F_2 = 1 = \phi^0$.  Now, since for arbitrary $n$ we know that $P\left(n-1\right)$
 and $P\left(n-2\right)$ are true, then $F_n = F_{n-1}+F_{n-2} \ge \phi^{n-4} + \phi^{n-3}
 = \phi^{n-4} \left(1 + \phi\right)$.  Since $\left(1 + \phi\right) = \phi^2$
 we have $F_n \ge \phi^{n-2}$, as claimed.

\vskip 0.1in
\noindent
{\bf Problem 10} Here we start with $n=10$, and indeed $2^{10} = 1024 > 10^3 = 1000$.
 So now, for arbitrary $n$, we know $2^{n-1} > \left(n-1\right)^3$.  If we multiply
 both sides by 2, we have $2^n > 2 \left(n-1\right)$.  So if $2 \left(n-1\right)^3 > n^3$
 for $n > 10$, then we have proved the relation.  Expanding shows that this requires
 that the expression $n^3 - 6 n^2 + 6n - 1$ is positive for $n > 10$ -- and in fact
 it is positive for all $n > {1 \over 2} \left( 5 + \sqrt{21} \right) \approx 4.7$.

\vskip 0.5in
\centerline{\tt Knuth 1.2.2}
\vskip 0.5in

\noindent
{\bf Problem 4} $0.125^{-2/3} = \left(1/8\right)^{-2/3} = \left(2^{-3}\right)^{-2/3} = 2^{2} = 4$.

\vskip 0.1in
\noindent
{\bf Problem 13} (a) This can be recast as proving that 
  $\left(1+x\right) \le \left(1 + x/n\right)^n$.
  Or, relabeling, $\left(1+n x \right) \le \left(1+x\right)^n$, where x is positive.  This
  is true for $n=1$.  Using induction, if $P\left(n\right)$ is true, then $\left(1+n x \right) \le
  \left(1+x\right)^n$.  Multiply both sides by the positive number $1+x$ to get
  $ \left( 1 + \left(n+1\right) x \right) < 1 + \left(n+1\right) x + n x^2 = 
  \left(1+n x\right) \left(1 + x\right) \le \left(1+x\right)^{n+1}$.  The outer parts are
  $P\left(n+1\right)$.
  
\vskip 0.1in
\noindent
{\bf Problem 15} Move $\log_b y$ across to the other side to give $\log_b x/y + \log_b y = \log_b x$.
 Now use the product rule (11): $\log_b xy = \log_b x + \log_b y$, and $x/y \times y = x$.

\vskip 0.1in
\noindent
{\bf Problem 19} We know that $\log_{10} n < 14$.  That means that $\log_2 n = \log_{10} n /
  \log_{10} 2 < \log_{10} n / 0.302 = 46.4$.  So it will fit in 47 binary digits.

\vskip 0.1in
\noindent
{\bf Problem 20} $\log_2 10 = 1 / \log_{10} 2$

\vskip 0.5in
\centerline{\tt Knuth 1.2.3}
\vskip 0.5in

\noindent
{\bf Problem 3} Because $0 \le n^2 \le 5$ is not a permutation.  For example, both $n=2$ and $n=-2$
 satisfy the relation.
 
\vskip 0.1in
\noindent
{\bf Problem 13} $\sum_{j=m}^{n} j = \sum_{j=0}^{n} - \sum_{j=0}^{m-1} = 1/2 \left( n \left(n+1\right) -
   m \left(m-1\right) \right)$.
   
\vskip 0.1in
\noindent
{\bf Problem 16} First, factor $x$ out: $\sum_{0 \le j \le n} j x^j = x \sum_{1 \le j \le n} j x^{j-1} =
 x \sum_{0 \le j \le n-1} \left(j + 1 \right) x^j$. But this can be written 
 $x \sum_{0 \le j \le n-1} j x^j + x \sum_{0 \le j \le n-1} x^j$ = $x \sum_{0 \le j \le n-1} j x^j - x n x^j
 + x \sum_{0 \le j \le n-1} x^j$.  Taking advantage of (14) from the text, $ \sum_{0 \le j \le n-1} =
  \left( { 1 - x^n } \over 1 - x \right)$, so we have 
  $$ \sum_{0 \le j \le n} j x^j = x \sum_{0 \le j \le n} j x^j - n x^{n+1} + x \left( { 1 - x^n} \over 1-x \right).$$
 This can be solved and rearranged to give
 $$ 
   \sum_{0 \le j \le n} j x^j = {n x^{n+2} - \left( n + 1 \right) x^{n+1} + x \over \left( 1 - x \right)^2}.
 $$

\vskip 0.5in
\centerline{\tt Knuth 1.2.4}
\vskip 0.5in

\noindent
{\bf Problem 2} The floor, $\lfloor x \rfloor$.

\vskip 0.1in
\noindent
{\bf Problem 5} $\lfloor x + {1 \over 2} \rfloor$.  This works the same way for negative
 values of $x$ {\it unless} $x$ is half integral (i.e., $x - \lfloor x \rfloor = 1/2$).

\vskip 0.1in
\noindent
{\bf Problem 14} The number 8 satisfies both equations, and we can see that adding
 any multiplier of 15 to both continues to satisfy them.  Therefore, the answer is 8.
 
\vskip 0.5in
\centerline{\tt Knuth 1.2.5}
\vskip 0.5in

\noindent
{\bf Problem 4} 2568 digits in base 10.  The leading digit is 4, since $\log_{10} 4 = 0.60206$.
 The last digit must be zero, since it has to be multiplied by 1000.  In fact,
 it should end in a large number of 0s, which could be computed from equation (8)
 for the prime factors $p=2$ and $p=5$.  In fact, since the multiplicity of 2 must be
 greater than that of 5, all we have to compute is the multiplicity of 5, which is 249,
 so $1000!$ must end in 249 zeros.

\vskip 0.1in
\noindent
{\bf Problem 6} $20! = 2^{18}\, 3^8\, 5^4\, 7^2\, 11\, 13\, 17\, 19$.  This one ends in
 4 zeros, and in fact is 2,432,902,008,176,640,000.

\vskip 0.1in
\noindent
{\bf Problem 9} We have $n! = n \Gamma\left(n\right)$, so $\sqrt{\pi}/2 = 1/2! = 1/2 \Gamma\left( 1/2 \right)$
 and $\Gamma\left(1/2\right) = \sqrt{\pi}$; this assumes exercise (10).  For the other part, we also have
 $n! = \Gamma\left(n+1\right)$, which implies $\Gamma\left(-1/2\right) = 1/2! = \sqrt{\pi}/2$.

\vskip 0.1in
\noindent
{\bf Problem 23} Start by computing $z \left(-z\right)! \, \Gamma \left( z \right)$.  The leading $z$ just cancels
 off the $1/z$ in (15), so this is just 
 $$
   z \left(-z\right)! \, \Gamma \left( z \right) =
      \left[ \lim_{m \to \infty} {m^{-z} m! \over \left(1-z\right)\left(2-z\right)
      \ldots\left(m-z\right)} \right] \left[ \lim_{m \to \infty} {m^{z} m! \over \left(1+z\right)\left(2-z\right)
      \ldots\left(m+z\right)} \right] .
 $$
 We can use the distributive law of products to combine these
 $$
    z \left(-z\right)! \, \Gamma \left( z \right) = \lim_{m \to \infty} m^0 m!\, m!\, \prod_{n=1}^{m}
       {1 \over \left( n-z \right)} {1 \over \left( n+z \right) } = \lim_{m \to \infty} m!\, m!\, \prod_{n=1}^{m}
        n^{-2} \left( 1 - z/n \right)^{-1} \left( 1 + z/n \right)^{-1}.
 $$
 We can again use the distributive law to break out the product 
 $$z \left(-z\right)! \, \Gamma \left( z \right)
  = \lim_{m \to \infty} m!\, m!\, \prod_{n=1}^{m} n^{-2} \prod_{n=1}^{m} \left(1 - z^2/n^2 \right)^{-1}.
 $$
 But $\prod_{n=1}^{m} n^{-1} = 1 / m!$, so the leading bits cancel to leave
 $z \left(-z\right)! \, \Gamma \left( z \right) = \prod_{n=1}^{\infty} \left(1 - z^2/n^2 \right)^{-1}$.
 Now we take advantage of $1 / \prod a_i = \prod 1 / a_i$ and the problem information that
  $\prod_{n=1}^{\infty} \left( 1 - z^2 / n^2 \right) = \sin \pi z / \pi z$ to finally reach
 $\left(-z\right)! \, \Gamma \left( z \right) = \pi / \sin \pi z$, as claimed.

\vskip 0.5in
\centerline{\tt Knuth 1.2.6}
\vskip 0.5in

\noindent
{\bf Problem 1} $n$.

\vskip 0.1in
\noindent
{\bf Problem 3} $52 \choose 13$, which is ${52! / 13! \, 39!} = 635,013,559,600.$

\vskip 0.1in
\noindent
{\bf Problem 7} Since the factorial is a much stronger function than the square,
 to minimize $k! \, \left(n-k\right)!$, $k = \lfloor n/2 \rfloor$, or
 $k = \lceil n/2 \rceil$.

\vskip 0.1in
\noindent
{\bf Problem 13} This is trivially shown by applying (9) repeatedly to the rightmost
 term of $r + n + 1 \choose n$ as follows:
 $$
   {r + n + 1 \choose n} = {r + n \choose n} + {r + n \choose n - 1} =
      {r + n \choose n} + {r + n - 1 \choose n - 1} + {r + n - 1 \choose n - 2} = \ldots.
 $$
 The final expanded sum is just $\sum_{k=0}^n {r+k \choose k}$.
    
\vskip 0.1in
\noindent
{\bf Problem 14} The program is to use the Stirling numbers to rewrite $k^4$ as binomials.
   Using table 2,
   $$k^4 = 24 {k \choose 4} + 36 {k \choose 3} + 22 {k \choose 2} + {k \choose 1}.$$
   We can now apply equation (11) to find $$\sum k^4 = 24 {n + 1 \choose 5} +
     36 { n + 1 \choose 4 } + 22 { n + 1 \choose 3 } + { n+1 \choose 2}.$$
   This in turn can be laboriously expanded by hand to give
   $\sum_{k=0}^n k^4 = n^5/5 + n^4/2 + n^3/3 - n/30.$
   
\vskip 0.1in
\noindent
{\bf Problem 18} Use (6) to rewrite ${r \choose m + k} = {r \choose r - m - k}$.  Then use
 (22) to rewrite
 $$ \sum_k {r \choose r - m - k} {s \choose n + k} = {r + s \choose r - m + k}. $$
   
\vskip 0.1in
\noindent
{\bf Problem 36} We can use the binomial theorem $\left(x+y\right)^n = \sum_{k} {n \choose k} x^k y^{n-k}$
 with $x = y = 1$ to see that $\sum_k {n \choose k} = 2^n$.  For the alternating sum, set $x=-1$ and $y=1$
 to find $\sum_k \left(-1\right)^{k} {n \choose k} = \delta_{n0}$.
 
\vskip 0.1in
\noindent
{\bf Problem 46} ${x + y \choose y} = {\left(x + y\right)! \over x! \, y!}$, and for large $n$, 
 $n! \simeq \sqrt{2 \pi n}\, n^n\, e^{-n}$.  Straightforward substitution gives 
 $${ x + y \choose y } \simeq \sqrt{ x + y \over 2 \pi \, x y } \left(x+y\right)^{x+y} x^{-x} y^{-y}.$$
 Now rewrite the last term as $\left(x+y\right)^{x} x^{-x} \left(x+y\right)^y y^{-y}$ and note that
 $\left(x+y\right)^x = x^x \left(1 + y/x\right)^x$ to finally get
 $$
   { x + y \choose y } \simeq \sqrt{ x + y \over 2 \pi \, x y } \left(1+y/x\right)^x \left(1+x/y\right)^y 
 $$
 and therefore ${2 n \choose n} \simeq 2^{2n} / \sqrt{\pi n} = 4^n / \sqrt{\pi n}$
   
\vskip 0.5in
\centerline{\tt Knuth 1.2.7}
\vskip 0.5in

\noindent
{\bf Problem 2} $$ H_{2^{m+1}} = H_{2^m} + {1 \over 2^m+1 } + {1 \over 2^m + 2 } + \ldots + {1 \over 2^{m+1}}
 \le H_{2^m} + {1 \over 2^m} + {1 \over 2^m} + \ldots + {1\over 2^m} = H_{2^m} + 1. $$

\vskip 0.5in
\centerline{\tt Knuth 1.2.8}
\vskip 0.5in

\noindent
{\bf Problem 2} $F_{1000} = {1 \over \sqrt{5}} \left( \phi^{1000} - \left(1-\phi \right)^{1000} \right) 
 \simeq {1 \over \sqrt{5}} \phi^{1000}$.  Using the logarithm to express this as $10$ to some power, 
  $F_{1000} \simeq 10^{1000 \log_{10} \phi - 1/2 \log_{10} 5} \simeq 10^{208.64}$.

\vskip 0.1in
\noindent
{\bf Problem 4} 0, 1, and 5.  For larger numbers, $F_n$ increases much faster than linearly, so these are
 all of them.
 
\vskip 0.5in
\centerline{\tt Knuth 1.2.9}
\vskip 0.5in

\noindent
{\bf Problem 1} First, it's the sum of the generating sequences for $2^n$ and $3^n$ by rule {\bf A}.  Next,
  use (5) and rule {\bf D} to note that the sequence for $c^n$ is just $1 / \left( 1-c z \right)$.  Therefore, the generating
  function for $2^n + 3^n$ is just $1 / \left( 1 - 2 z \right) + 1 / \left(1 - 3 z\right)$.

\vskip 0.1in
\noindent
{\bf Problem 6} We can use the product rule {\bf C} to see that the generating function for this sequence
 is just the product of the generating functions for $1/k$ and $1/k$.  But this is just (17), $ \log \left(1-z\right)^{-1}$.
 Therefore, the generating function for this sequence is just 
 $$
    G\left(z\right) = \left( \log {1 \over 1 - z} \right)^2.
 $$
 The derivative is just 
 $$
   G^{\prime} \left(z\right) = {2 \over 1-z} \log {1 \over 1-z}.
 $$
 But this is twice the generating function for the harmonic numbers of (18), or
 $G^{\prime} = 2 \sum H_k z^k$.

\vskip 0.5in
\centerline{\tt Knuth 1.2.10}
\vskip 0.5in

\noindent
{\bf Problem 1} 
  $$ 
     p_{n0} = {1 \over n} p_{\left(n-1\right)\left(k-1\right)} + {n-1 \over n} p_{\left(n-1\right) k}
            = {n - 1 \over n} p_{\left(n-1\right) 0}
  $$
  because $p_{nk} = 0$ for $k < 0$.  Since $p_{10} = 1$, this is $p_{n0} = \prod {n - 1 \over n} = 
    {\left(n-1\right)! \over n!} = {1 \over n}$.  Thus, $G\left(0\right) = 1/n$, and so the probability
  that $X\left[n\right]$ is the maximum is just $1/n$.

\vskip 0.1in
\noindent
{\bf Problem 2} $G^{\prime\prime} \left(z\right)= \sum_k k \left(k - 1\right) p_k z^{k-2}$ so
 $G^{\prime\prime} \left(1\right) = \sum_k k^2 p_k - \sum_k k p_k.$  The last term is just 
 $G^{\prime} \left(1\right)$, so $\sum_k k^2 p_k = G^{\prime\prime} \left(1\right) + G^{\prime}
  \left(1\right).$  Now using (10), we immediately recover the desired equation.
  
\vskip 0.1in
\noindent
{\bf Problem 8} There are $M$ ways to chose the first, $M-1$ the second, etc.  Together, there are
 $M^{\underline n}$ ways to do this.  Since there are $M^n$ total possibilities, the probability
 that a random one satisfies the requirement is $M^{\underline n} / M^n$.
 
\vskip 0.5in
\centerline{\tt Knuth 1.3.1$^{\prime}$}
\vskip 0.5in

\noindent
{\bf Problem 1} $\left(1\ 1\ 1\ 1\ 1\ 0\ 1\ 1\ 0\ 0\ 1\right)_2$ = $\left(7\ d\ 9 \right)_{16}$ = 
 $\left(3\ 7\ 3\ 1\right)_8$ = $\left(2\ 0\ 0\ 9\right)_{10}$.
 
 \vskip 0.1in
 \noindent
 {\bf Problem 2} As hexidecimal digits, the odd ones are B, D, F (and b, d, f).
  As ASCII, A, C, E, a, c, e.

\vskip 0.1in
\noindent
{\bf Problem 5} Prove that $x \neq s\left(\alpha\right)  \Rightarrow x \not\equiv u\left(\alpha\right)$.
Note that $-2^{n-1} \le s\left(\alpha\right) \ge 2^{n-1}-1$.  Therefore, if
if $-2^{n-1} \le x < 2^{n-1}$ then we must have $-2^n < x - s\left(\alpha\right) < 2^n$.
This means that if $x \neq s\left(\alpha\right)$, then we must have
$x \not\equiv s\left(\alpha\right)$ (mod $2^n$).  Next we need to relate 
$s\left(\alpha\right)$ and $u\left( \alpha \right)$.  The relation is
$s\left(\alpha\right) = u\left(\alpha\right)$ if $s_{63} \neq 1$ ($s_{63}$ is the first
bit), and $s\left(\alpha\right) = u\left(\alpha\right)-2^n$ if $s_{63} = 1$.
In the first case, clearly $s\left(\alpha\right) \equiv u\left(\alpha\right)$ and
in the second $s \left( \alpha \right) = u\left(\alpha\right) - 2^n \equiv u \left(\alpha\right)$.
So $s\left( \alpha \right) \equiv u\left(\alpha\right)$, and by the transitivity of
the modulo equivalence, $x \not\equiv u\left(\alpha\right)$.  On the other hand,
if $x = s \left(\alpha\right)$, then clearly $x \equiv s \left( \alpha \right) \equiv u \left( \alpha \right)$.

\vskip 0.1in
\noindent
{\bf Problem 6} Negating a $n$ bit number is $u\left( \overline \alpha \right) = 2^n - 1 - 
u \left( \alpha \right)$.  Therefore $u \left( \overline \alpha \right) + 1 \equiv - u \left( \alpha \right)$
(mod $2^n$).  From Problem 5, this then implies that $s \left( \overline \alpha \right) + 1 =
- s \left( \alpha \right)$.  This fails if $s \left( \overline \alpha \right) + 1 = 2^n$.

\vskip 0.1in
\noindent
{\bf Problem 7} Yes -- since these are the high halves of the slots, there is no
concern about extending the sign bit which could make this not work.

\vskip 0.1in
\noindent
{\bf Problem 10} The maximum positive integer is $2^n-1$, the maximum negative
is $2^n$, so if we set $\$Y = -2^n$, $\$Z=-1$.

\vskip 0.1in
\noindent
{\bf Problem 11} From {\bf Problem 5}, $s\left(\$Y\right) \equiv u\left(\$Y\right)$
 and $s\left(\$Z\right) \equiv u\left(\$Z\right)$ (both mod $2^{64}$).  This implies
 that $s\left(\$Y\right) s\left(\$Z\right) \equiv u\left(\$Y\right) u\left(\$Z\right)$.
 As long as there is no overlow, then, this is true.  With overflow it isn't clear, 
 since the spec for {\bf MULU} doesn't specify what happens on overflow.
 
\vskip 0.1in
 \noindent
 {\bf Problem 12} The critical point is that if their {\it is} carry, then 
 \$X will be less than \$Y after the {\bf ADDU}.  Therefore, if we want
 the carry bit to end up in \$A, {\bf ADDU \$X, \$Y, \$Z, CMPU \$A, \$X, \$Y}.
 Then \$A will be positive if there is carry, negative or 0 if not.

\vskip 0.1in
\noindent
{\bf Problem 13} Overflow occurs if \$Y and \$Z have the same sign
but their sum has the opposite sign.  So we need to find a way to
effectively store the sign before and after addition.  Start with
{\bf NXOR \$0, \$Y, \$Z}.  \$0 will have a 1 as the first bit if they
have the same sign.  Then  {\bf ADDU \$X, \$Y, \$Z, XOR \$1, \$X, \$Y}.
Now \$1 will have a 0 as the first digit if the sum has the same sign 
as the inputs, 1 if different.  We can finally test for overflow by doing 
{\bf AND \$1, \$0, \$0}.  This will have a 1 as the first bit if overflow
has occured, so it will be negative as a signed quantity.  So finally
{\bf CSN carry, \$1, 0}.
 
\vskip 0.1in
 \noindent
{\bf Problem 20} {\bf 4ADDU \$Y, \$X, \$X, 4ADDU \$X, \$Y, \$Y}.

\vskip 0.1in
\noindent
{\bf Problem 22} The code works unless there is overflow.  To avoid
this problem, {\bf CMP \$0, \$1, \$2, BN \$0, case1}.  

\vskip 0.1in
\noindent
{\bf Problem 23} Replace {\bf BN} with {\bf BNP}.

\vskip 0.1in
\noindent
{\bf Problem 25} {\bf XOR \$Z, \$X, \$Y} gives us a 1 at every
position where the two vectors are different.  Following this 
by {\bf SADD \$1, \$Z, 0} puts the Hamming distance in \$1.

\vskip 0.1in
\noindent
{\bf Problem 26} The saturating subtraction operator \.{-} on 
individual bits must satisfy 0 \.{-} 0 = 0, 0 \.{-} 1 = 0, 1 \.{-} 0 = 1,
1 \.{-} 1 = 0.  Since only one is non-zero, the logical place
to start is with some variance of the and operator.  Examining
(8) in 1.3.1, it can be done with the and operator if we not
the second argument -- therefore, what we want is {\bf ANDN}.

\vskip 0.1in
\noindent
{\bf Problem 27} {\bf BDIF \$W, \$Y, \$Z}.  If $y > z$, then this is
$y-z$, if $y < z$ this is zero.  So we add $z$ back in to get
either $y$ or $z$, whichever is larger. {\bf ADDU \$X, \$W, \$Z}
will store the maximum in \$X.  We don't have to worry about overflow
here.  To get the minimum we simply replace addition with subtraction:
{\bf SUBU \$W, \$Y \$W} -- so, if $y>z$ this is $y-\left(y-z\right)=z$, otherwise
it's $y-0=y$.

\vskip 0.1in
\noindent
{\bf Problem 30} Counting the number of bits that agree is an issue
because we will get false positives of any bye that begins with \#2.
So the approach is to count the number of bits that differ, then use
one of the bytewise operators to convert this to the number of bytes
that agree, and then add those.  So, the first step is 
 {\bf XOR \$1, \$0, [2020202020202020]}.  This puts a one
everywhere that they don't match and a zero where they do.  
Now we use byte wise arithmetic {\bf BDIF \$1, [0101010101010101],\$1}.
This puts a 1 in every byte that differs in zero places, and a 0 in every
byte that differs in one or more place.  Now, finally, {\bf SADD \$1,\$1, 0}
does the counting.

\vskip 0.5in
\centerline{ \tt Knuth 1.3.2$^{\prime}$ }
\vskip 0.5in

\noindent
{\bf Problem 1} a) it refers to line 24, the previous 4H. b)  No.  Lines 23 and 31 would now do the wrong thing.

\end