\topglue 0.5in
\centerline{\tt Knuth 7.1.1}
\vskip 0.5in

\noindent
{\bf Problem 3} (a) max corresponds to logical or: $\vee$ (b) min corresponds to 
logical and: $\wedge$ (c) $-x$ is negation or right complementation: $\overline{x}$
(d) $x \cdot y$ is equivalence: $x \equiv y$.

\vskip 0.1in
\noindent
{\bf Problem 55} Two different ways to write this: $n_1 \times n_2 +
n_1 \times n_3 + \ldots + n_1 \times n_k + n_2 \times n_3 + \ldots + 
n_{k-1} \times n_k$.  Or, alternatively, start with the complete graph
and subtract all the edges between vertices in the same components:
${N \choose 2} - {n_1 \choose 2} - {n_2 \choose 2} - \ldots - {n_k \choose 2}$
(where $N = \sum_i n_i$).

\vskip 0.1in
\noindent
{\bf Problem 56} True.  Simple means that there are not multiple arcs between
the same vertices in the digraph.  That rules out self edges, since converting
a multigraph to a digraph would add two arcs for each self loop in the multigraph.
Therefore, the multigraph can't have any self loops, and is a graph.

\vskip 0.1in
\noindent
{\bf Problem 57} This would be true if the question asked about strongly
connected components, but it doesn't.  So false.  A trivial example is to
have two of three vertices point into the third vertex.
\end